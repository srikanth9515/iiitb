\begin{figure}[!ht]
	\begin{center}
			\resizebox{\columnwidth}{!}{\input{./figs/triangle/tri_sss.tex}}
	\end{center}
	\caption{Area of a Triangle}
	\label{fig:tri_sss}	
\end{figure}
\iffalse
\begin{figure}[!ht]
\centering
\resizebox{\columnwidth}{!}{\input{./figs/triangle/tri_right_angle.tex}}
\caption{}
\label{fig:tri_right_angle_area}	
\end{figure}
\fi
\renewcommand{\theequation}{\theenumi}
\begin{enumerate}[label=\thesection.\arabic*.,ref=\thesection.\theenumi]
\numberwithin{equation}{enumi}

\item
\label{prob:tri_area_sin}
	Show that the area of $\Delta ABC$ in Fig. 	\ref{fig:tri_sss}	is $\frac{1}{2}ab \sin C$.

\solution We have
%
\begin{equation}
ar\brak{\Delta ABC} = \frac{1}{2}ah = \frac{1}{2}ab\sin C \quad \brak{\because \quad h = b \sin C}.
\label{eq:tri_area_sin}
\end{equation}

\item
	Show that 
	\begin{equation}
	\frac{\sin A}{a} = \frac{\sin B}{b} = \frac{\sin C}{c}
	\end{equation}

\solution Fig. \ref{fig:tri_sss} can be suitably modified to obtain 
\begin{align}
ar\brak{\Delta ABC} = 
\frac{1}{2}ab\sin C = \frac{1}{2}bc\sin A = \frac{1}{2}ca\sin B
\end{align}
Dividing the above by $abc$, we obtain
	\begin{equation}
\label{eq:tri_sin_form}
	\frac{\sin A}{a} = \frac{\sin B}{b} = \frac{\sin C}{c}
	\end{equation}
This is known as the sine formula.	
%
%
\item Show that 
%
\begin{align}
\label{eq:trig_id_sin_inc}
\alpha > \beta \implies \sin \alpha > \sin \beta
\end{align}
%

\begin{figure}[!ht]
	\begin{center}
		
		%\includegraphics[width=\columnwidth]{./figs/fig:tri_sin_inc}
		%\vspace*{-10cm}
		\resizebox{\columnwidth}{!}{\input{./figs/triangle/tri_sin_inc.tex}}
	\end{center}
	\caption{}
	%\caption{$\sin \brak{\theta_1+\theta_2} = \sin\theta_1\cos\theta_2 + \cos\theta_1\sin\theta_2$}
	\label{fig:tri_sin_inc}	
\end{figure}
\solution In Fig. \ref{fig:tri_sin_inc}, 	
%
\begin{align}
ar\brak{\triangle ABD} &< ar \brak{\triangle ABC}
\\
\implies \frac{1}{2}lc \sin \theta_1 &<  \frac{1}{2}ac \sin \brak{\theta_1 + \theta_2 }
\\
\implies \frac{l}{a} &< \frac{\sin \brak{\theta_1 + \theta_2 }}{\sin \theta_1}
\\
\text{or, } 1 < \frac{l}{a} &< \frac{\sin \brak{\theta_1 + \theta_2 }}{\sin \theta_1}
\\
\implies \frac{\sin \brak{\theta_1 + \theta_2 }}{\sin \theta_1} > 1
\end{align}
%
from Theorem \ref{them:hyp_largest}. This proves \eqref{eq:trig_id_sin_inc}.
%From \eqref{eq:trig_id_sum_diff3},
%%
%\begin{multline}
% \sin \theta_1 - \sin \theta_2 = 2\sin\brak{\frac{\theta_1-\theta_2}{2}}
%\\
%\times \cos\brak{\frac{\theta_1+\theta_2}{2}} > 0, \because \theta_1-\theta_2 > 0
%\end{multline}
%
\item
	Using 
	\figref{fig:tri_sin_inc},
%Fig. \ref{trig_id_sin_theta}, 
show that 
	%
\begin{equation}
\label{trig_id_sin_theta_eq}
\sin  \theta_1 = \sin \brak{\theta_1 + \theta_2}\cos \theta_2 - \cos\brak{\theta_1+\theta_2}\sin\theta_2
\end{equation}	
	%
\iffalse
\begin{figure}[!ht]
	\begin{center}
		
		%\includegraphics[width=\columnwidth]{./figs/trig_id_sin_theta}
		%\vspace*{-10cm}
		\resizebox{\columnwidth}{!}{\input{./figs/triangle/tri_sin_inc.tex}}
	\end{center}
	\caption{$\sin \brak{\theta_1+\theta_2} = \sin\theta_1\cos\theta_2 + \cos\theta_1\sin\theta_2$}
	\label{trig_id_sin_theta}	
\end{figure}
\fi
%

\solution The following equations can be obtained from the figure using the forumula for the area of a triangle
%
\begin{align}
ar \brak{\Delta ABC} &= \frac{1}{2}ac \sin\brak{\theta_1 + \theta_2} \\
&= ar \brak{\Delta BDC} + ar \brak{\Delta ADB} \\
&= \frac{1}{2}cl \sin{\theta_1} + \frac{1}{2}al \sin{\theta_2} \\ 
&= \frac{1}{2}ac \sin{\theta_1} \sec \theta_2 + \frac{1}{2}a^2 \tan{\theta_2} 
\end{align}
$\brak{\because
	l = a \sec \theta_2}$.  From the above,
\begin{align}
\sin\brak{\theta_1 + \theta_2} &=  \sin{\theta_1} \sec \theta_2 + \frac{a}{c} \tan{\theta_2} \\
	&=  \sin{\theta_1} \sec \theta_2 
+ \cos\brak{\theta_1 + \theta_2} \tan{\theta_2} 
\end{align}
Multiplying both sides by $\cos \theta_2$,
\begin{align}
\sin\brak{\theta_1 + \theta_2}\cos{\theta_2} =  \sin{\theta_1}  
+ \cos\brak{\theta_1 + \theta_2} \sin\theta_2  
\end{align}
%
resulting in
\eqref{trig_id_sin_theta_eq}.
\item Find Hero's formula for the area of a triangle.
\\
\solution 
%In Fig. \ref{fig:rt_triangle}, from Baudhayana's theorem, 
%\begin{align}
%\label{eq:tri_geo_baudh}
%b^2 = a^2+c^2 &
%\\
%=b^2\cos^2C+b^2\sin^2C &
%\\
%\implies \cos^2C+\sin^2C &= 1
%\end{align}
%
%In Fig. \ref{fig:tri_const_ex_cos_form}, 
From \eqref{prob:tri_area_sin}, the area of $\triangle ABC$ is 
{\footnotesize
\begin{align}
\label{eq:tri_geo_area_sin_form}
 \frac{1}{2}ab\sin C
%\\
&=\frac{1}{2}ab\sqrt{1-\cos^2C} 
\quad \brak{\text{from } \eqref{eq:tri_sin_cos_id}
%\eqref{eq:tri_geo_baudh}
}
\\
&=\frac{1}{2}ab\sqrt{1-\brak{\frac{a^2+b^2-c^2}{2ab}}^2} \brak{\text{from } \eqref{eq:tri_cos_form}
}
\\
&=\frac{1}{4}\sqrt{\brak{2ab}^2-\brak{a^2+b^2-c^2}}
\\
&=\frac{1}{4}\sqrt{\brak{2ab+a^2+b^2-c^2}\brak{2ab-a^2-b^2+c^2}}
\\
&= \frac{1}{4}\sqrt{\cbrak{\brak{a+b}^2-c^2}\cbrak{c^2-\brak{a-b}^2}}
\\
&= \frac{1}{4}\sqrt{\brak{a+b+c}\brak{a+b-c}\brak{a+c-b}\brak{b+c-a}}
\label{eq:tri_ex_hero_temp}
\end{align}
}
Substituting 
%
\begin{align}
s=\frac{a+b+c}{2}
\end{align}
%
in \eqref{eq:tri_ex_hero_temp}, the area of $\triangle ABC$ is 
%
\begin{align}
\label{eq:tri_area_hero}
\sqrt{s\brak{s-a}\brak{s-b}\brak{s-c}}
\end{align}
%
This is known as Hero's formula.
\iffalse
\item In a triangle, the side opposite the greater angle is greater.
\begin{figure}[!ht]
	\begin{center}
			\resizebox{\columnwidth}{!}{\input{./figs/quad/tri_ang_side.tex}}
	\end{center}
	\caption{Side opposite the greater angle is greater}
	\label{fig:tri_ang_side}	
\end{figure}
\\
\solution In Fig. 	\ref{fig:tri_ang_side},	let
%
\begin{align}
\angle B > \angle C
\end{align}
%
Then, using the sine formula,
%
\begin{align}
\frac{\sin B}{b} &=\frac{\sin C}{c}
\\
\implies   \frac{\sin B}{\sin C} &= \frac{b}{c} > 1
\end{align}
using \eqref{eq:trig_id_sin_inc}.


\fi
\end{enumerate}
