%%
\renewcommand{\theequation}{\theenumi}
\begin{enumerate}[label=\thesection.\arabic*.,ref=\thesection.\theenumi]
\numberwithin{equation}{enumi}

\item In Fig. \ref{fig:tri_ccentre}, points $\vec{A}, \vec{B}, \vec{C}$  are at a distance $R$ from $\vec{O}$.  Trace all such points. The locus of such points is defined as a {\em circle}.
%
	\iffalse
\\
\solution This is done by the following python code
%
\begin{lstlisting}
codes/circle/tri_ccircle.py
\end{lstlisting}
%
and the equivalent latex-tikz code to draw Fig. \ref{fig:tri_ccircle} is
%
\begin{lstlisting}
figs/triangle/tri_ccircle.tex
\end{lstlisting}
\fi
\begin{figure}[!ht]
	\begin{center}
		
		\resizebox{\columnwidth}{!}{\input{./figs/circle/tri_ccircle.tex}}
	\end{center}
	\caption{Circumcircle of $\triangle ABC$}
	\label{fig:tri_ccircle}	
\end{figure}
%
\item Line segements joining any two points on the circle are defined to be {\em chords}. The sides of $\triangle ABC$ are chords of the circle in 	\eqref{fig:tri_ccircle}	

\item From  \eqref{prob:tri_perp_bisect}, it is established that the line segment joining the centre of a circle to the mid point of a chord bisects the chord.
\item From \eqref{prob:tri_ccentre_subtend}, it is clear that the angle subtended by a chord at the centre of the circle is twice the angle subtended at any point on the circle.
\label{them:tri_ccentre_subtend}
%

\end{enumerate}

