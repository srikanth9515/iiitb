\renewcommand{\theequation}{\theenumi}
\begin{enumerate}[label=\arabic*.,ref=\thesubsection.\theenumi]
\numberwithin{equation}{enumi}
\item $ABCD$ is a cyclic quadrilateral in which $AC$ and $BD$ are its diagonals. If $\angle DBC = 55\degree$ and $\angle BAC = 45\degree$, find $\angle BCD$
%
\item Two circles of radii 5 cm and 3 cm intersect at two points and the distance between their centres is 4 cm. Find the length of the common chord.
%
%
\item  A,B and C are three points on a circle with centre $O$ such that $\angle BOC = 30\degree $ and $ \angle AOB = 60\degree$. If D is a point on the circle other than the arc ABC, find $\angle ADC$.
%
\item $ \angle PQR = 100\degree$, where $P, Q$ and R are
points on a circle with centre $O$. Find $\angle OPR$.
\item $A, B, C, D$ are points on a circle such that $ \angle ABC = 69\degree, \angle ACB = 31\degree$, find
$\angle BDC$.
\item $A, B, C$ and $D$ are four points on a
circle. $AC$ and $BD$ intersect at a point $E$ such
that $\angle BEC = 130\degree$ and $\angle ECD = 20\degree$. Find $\angle BAC$.
\item $ABCD$ is a cyclic quadrilateral whose diagonals intersect at a point $E$. If $\angle DBC = 70\degree,
\angle BAC$ is $30\degree$, find $\angle BCD$. Further, if $AB = BC$, find $\angle ECD$.
%
\item Two chords $AB$ and $CD$ of lengths 5 cm and 11 cm respectively of a circle are parallel
to each other and are on opposite sides of its centre. If the distance between $AB$ and
$CD$ is 6 cm, find the radius of the circle.
\item The lengths of two parallel chords of a circle are 6 cm and 8 cm. If the smaller chord is
at distance 4 cm from the centre, what is the distance of the other chord from the
centre?
%
\item A tangent $PQ$ at a point $P$ of a circle of radius 5 cm meets a line through the centre $O$ at a point $Q$ so that $OQ =$ 12 cm. Find the length of $PQ$.
%
\item $PQ$ is a chord of length 8 cm of a circle of radius 5 cm. The tangents at $P$ and $Q$ intersect at a point $T$. Find the length $TP$.
%
\item From a point $Q$, the length of the tangent to a circle is 24 cm and the distance of $Q$ from the centre is 25 cm. Find the radius of the circle is 
\item  If $TP$ and $TQ$ are the two tangents to a circle with centre $O$ so that  $\angle  POQ = 110 \degree $, then find  $\angle  PTQ$
\item  If tangents $PA$ and $PB$ from a point $P$ to a circle with centre $O$ are inclined to each other at angle of 80 $\degree$ , then find  $\angle  POA $
%
\item The length of a tangent from a point $A$ at distance 5 cm from the centre of the circle is 4 cm. Find the radius of the circle.
\item  Two concentric circles are of radii 5 cm and 3 cm. Find the length of the chord of the larger circle which touches the smaller circle.
%
\item A $\triangle ABC$ is drawn to circumscribe a circle of radius 4 cm such that the segments $BD$ and $DC$ into which $BC$ is divided by the point of contact $D$ are of lengths 8 cm and 6 cm respectively. Find the sides $AB$ and $AC$.
%
	\iffalse
\item The cost of fencing a circular field at the rate of \rupee 24 per metre is \rupee 5280. The field is to be ploughed at the rate of \rupee 0.50 per $m^2$.  Find the cost of ploughing the field.	
	\fi
\item The radii of two circles are 19 cm and 9 cm respectively. Find the radius of the circle which has circumference equal to the sum of the circumferences of the two circles.
\item The radii of two circles are 8 cm and 6 cm respectively. Find the radius of the circle having area equal to the sum of the areas of the two circles.
\item A circular  archery target is marked with its five scoring regions from the centre outwards as Gold, Red, Blue, Black and White. The diameter of the region representing Gold score is 21 cm and each of the other bands is 10.5 cm wide. Find the area of each of the five scoring regions.
\item The wheels of a car are of diameter 80 cm each. How many complete revolutions does each wheel make in 10 minutes when the car is travelling at a speed of 66 km per hour?
%
 \item Find the area of the sector of a circle with radius 4 cm and of angle 30 $\degree$ . Also, find the area of the corresponding major sector.
\item Find the area of the segment $AYB$, if radius of the circle is 21 cm and
 $\angle  AOB = 120 \degree$ .
%
\item Find the area of a sector of a circle with radius 6 cm if angle of the sector is 60 $\degree$ . 
\item Find the area of a quadrant of a circle whose circumference is 22 cm. 3. The length of the minute hand of a clock is 14 cm. Find the area swept by the minute hand in 5 minutes.
\item A chord of a circle of radius 10 cm subtends a right angle at the centre. Find the area of the corresponding : 
\begin{enumerate}
\item minor segment 
\item major sector.
\end{enumerate}

\item In a circle of radius 21 cm, an arc subtends an angle of 60 $\degree$  at the centre. Find: 
\begin{enumerate}
\item the length of the arc 
\item area of the sector formed by the arc 
\item area of the segment formed by the corresponding chord
\end{enumerate}
\item A chord of a circle of radius 15 cm subtends an angle of 60 $\degree$  at the centre. Find the areas of the corresponding minor and major segments of the circle. 
\item A chord of a circle of radius 12 cm subtends an angle of 120 $\degree$  at the centre. Find the area of the corresponding segment of the circle. 
\item A horse is tied to a peg at one corner of a square shaped grass field of side 15 m by means of a 5 m long rope. Find 
\begin{enumerate}
\item the area of that part of the field in which the horse can graze.
\item the increase in the grazing area if the rope were 10 m long instead of 5 m.
\end{enumerate}
\item A brooch is made with silver wire in the form of a circle with diameter 35 mm. The wire is also used in making 5 diameters which divide the circle into 10 equal sectors. Find : 
\begin{enumerate}
\item the total length of the silver wire required. 
\item the area of each sector of the brooch
\end{enumerate}
\item An umbrella has 8 ribs which are equally spaced. Assuming umbrella to be a flat circle of radius 45 cm, find the area between the two consecutive ribs of the umbrella.
\item A car has two wipers which do not overlap. Each wiper has a blade of length 25 cm sweeping through an angle of 115 $\degree$ . Find the total area cleaned at each sweep of the blades.
\item  To warn ships for underwater rocks, a lighthouse spreads a red coloured light over a sector of angle 80 $\degree$  to a distance of 16.5 km. Find the area of the sea over which the ships are warned.
	\iffalse
\item  A round table cover has six equal designs. If the radius of the cover is 28 cm, find the cost of making the designs at the rate of \rupee 0.35 per $cm^2$
. 
\fi
%
\item Two circular flower beds are located on opposite sides of a square lawn $ABCD$ of side 56 m. If the centre $O$f each circular flower bed is the point of intersection O of the diagonals of the square lawn, find the sum of the areas of the lawn and the flower beds.
%
\item Four circles are inscribed  inside a square $ABCD$ of side 14 cm such that each one touches exernally two adjacent sides of the square and two  other circles.  Find the region between the circles and the square.
\item  $ABCD$ is a square of side 10 cm and semicircles are drawn with each side of the square as diameter. Find the area enclosed by the circular arcs.
%
\item P is a point on the semi-circle formed with diameter $QR$. Find the area between the semi-circle and $\triangle PQR$ if $PQ$ = 24 cm, PR = 7 cm and O is the centre $O$f the circle.
\item $AC$ and $BD$ are two arcs on concentric circles with radii 14 cm and 7 cm respectively, such that $\angle AOC = 40\degree$.  Find the area of the region $ABDC$.
%
\item Find the area between a square $ABCD$ of side 14cm and the semi circles $APD$ and $BPC$.
\item Find the area of the  region enclosed by  a circular arc of radius 6 cm drawn with vertex $O$ of an equilateral triangle OAB of side 12 cm as centre.
%
\item From each corner of a square of side 4 cm a quadrant of a circle of radius 1 cm is cut and also a circle of diameter 2 cm is cut. Find the area of the remaining portion of the square.\item In a circular table cover of radius 32 cm, a design is formed leaving an equilateral $\triangle ABC$ in the middle. Find the area of the design.
%
\item $ABCD$ is a square of side 14 cm. With centres A, B, C and D, four circles are drawn such that each circle touches externally two of the remaining three circles. Find the area within the square that lies outside the circles.
\item The left and right ends of a racing track are semicircular.
The distance between the two inner parallel line segments is 60 m and they are each 106 m long. If the track is 10 m wide, find : 
\begin{enumerate}
\item the distance around the track along its inner edge 
\item the area of the track.
\end{enumerate}
\item $AB$ and $CD$ are two diameters of a circle (with centre $O$) perpendicular to each other and OD is the diameter of a  smaller circle inside. If $OA$ = 7 cm, find the area of the smaller circle.
\item The area of an equilateral $\triangle ABC$ is 17320.5 $cm^2$
. With each vertex of the triangle as centre, a circle is drawn with radius equal to half the length of the side of the triangle. Find the area of region within the triangle but outside the circles. 
\item On a square handkerchief, nine circular designs are inscribed touching each other, each of radius 7 cm. Find the area of the remaining portion of the handkerchief.
\item $OACB$ is a quadrant of a circle with centre $O$ and radius 3.5 cm. $D$ is a point on $OA$.  If OD = 2 cm, find the area of the
\begin{enumerate}
\item quadrant $OACB$,
 \item the region between the quadrant and $\triangle OBD$.
\end{enumerate}
\item A square $OABC$ is inscribed in a quadrant $OPBQ$. If $OA$ = 20 cm, find the area between the square and the quadrant.
\item $AB$ and $CD$ are respectively arcs of two concentric circles of radii 21 cm and 7 cm and centre $O$.  If  $\angle  AOB = 30 \degree$ , find the area of the region $ABCD$.
\item ABC is a quadrant of a circle of radius 14 cm and a semicircle is drawn with $BC$ as diameter. Find the area of the crescent formed.
\item Find the area common between the two quadrants of circles of radius 8 cm each if the centres of the circles lie on opposite sides of a square.
\item Find the area of the sector of a circle with radius 4 cm and of angle 30$\degree$. Also, find the area of the corresponding major sector.
\item A pole has to be erected at a point on the boundary of a circular park of diameter 13 metres in such a way that the differences of its distances from two diametrically opposite fixed gates A and B on the boundary is 7 metres. Is it possible to do so? If yes, at what distances from the two gates should the pole be erected?
%
\item Draw a triangle whose sides are 8cm and 11cm and the perimeter is 32 cm and find its area.
%
%\solution Use \eqref{eq:tri_hero}.
%
	\iffalse
\item A triangular park $ABC$ has sides 120m, 80m and 50m. A gardener Dhania has to put a fence all around it and also plant grass inside. Draw this park.  How much area does she need to plant? Find the cost of fencing it with barbed wire at the rate of \rupee 20 per metre leaving a space 3m wide for a gate on one side.
	\fi
%\\
%\solution Use \eqref{eq:tri_hero}.
%
\item The sides of a triangular plot are in the ratio of 3 : 5 : 7 and its perimeter is 300 m. Draw the plot and find its area.
%\\
%\solution Use \eqref{eq:tri_hero}.
\item A tower stands vertically on the ground.  From a point on the ground, which is 15m away from the foot of the tower, the angle of elevation of the top of the tower is found to be 60$\degree$.  Find the height of the tower.
%
%\begin{figure}[!ht]
%\includegraphics[width=\columnwidth]{./figs/Trig/pg1.eps}
%\caption{}
%\label{fig:trig_pg1}
%\end{figure}
%%
%\\
%\solution Fig. \ref{fig:trig_pg1} summarizes the problem. 
%%
%\begin{align}
%h = b\tan\theta = 15\tan60\degree = 15\sqrt{3}
%\end{align}
%
\item An electrician has to repair an electric fault pole of height 5m.  She needs to reach a point 1.3m below the top of the pole to undertake the repair work.  What should be the length of the ladder that she should use which, when inclined at an angle of 60$\degree$ to the horizontal, would enable her to reach the required position?  Also, how far from the foot of the pole should she place the foot of the ladder?
%
%\begin{figure}[!ht]
%\includegraphics[width=\columnwidth]{./figs/Trig/pg2.eps}
%\caption{}
%\label{fig:trig_pg2}
%\end{figure}
%%
%\\
%\solution Fig. \ref{fig:trig_pg2} summarizes the problem. The objective is to find $l$ and $b$.  From the figure,
%%
%if 
%\begin{align}
%\cot \theta &=\frac{1}{\tan \theta},
%\\
%h-x &= l\sin \theta = b\tan \theta
%\\
%\implies l &= \brak{h-x}\csc \theta = 3.7\csc60\degree 
%\\
%\text{and } b&=\brak{h-x}\cot\theta = 3.7 \cot \degree 
%\end{align}
\item An observer 1.5m tall is 28.5m away from a chimney.  The angle of elevation of the top of the chimney from her eyes is 45$\degree$.  What is the height of the chimney?
%
%
%\begin{figure}[!ht]
%\includegraphics[width=\columnwidth]{./figs/Trig/pg3.eps}
%\caption{}
%\label{fig:trig_pg3}
%\end{figure}
%%
%\\
%\solution Fig. \ref{fig:trig_pg3} summarizes the problem. The objective is to find $h$.  From the figure,
%%
%\begin{align}
%h-h_1 &=  b\tan \theta
%\\
%\implies h &= h_1+b\tan\theta 
%\\
%&= 1.5+28.5\tan45\degree 
%\\
%&= 30m
%\end{align}
\item From a point $\vec{P}$ on the ground the angle of elevation of the top of a 10m tall building is 30$\degree$.  A flag is hoisted at the top of the building and the angle of elevation of the top of the flagstaff from $\vec{P}$ is $45\degree$.  Find the length of the flagstaff and the distance of the building from the point $\vec{P}$.
%
%\begin{figure}[!ht]
%\includegraphics[width=\columnwidth]{./figs/Trig/pg4.eps}
%\caption{}
%\label{fig:trig_pg4}
%\end{figure}
%%
%\\
%\solution Fig. \ref{fig:trig_pg4} summarizes the problem. The objective is to find $h_2$ and $b$ while $h_1$ is known.  From the figure, 
%%
%\begin{align}
%h_1+h_2 &=  b\tan \theta_1
%\\
%h_1 &= b\tan \theta_2
%\end{align}
%%
%This can be expressed as the matrix equation 
%%
%\begin{align}
%\myvec{
%\tan \theta_1 & -1
%\\
%\tan \theta_2 &0
%}\myvec{b\\h_2}
%= h_1\myvec{1\\1}
%\end{align}
%%
%and solved.
\item The shadow of a tower standing on a level ground is found to be 40m longer when the Sun's altitude is 30$\degree$ than when it is $60\degree$.  Find the height of the tower.
%
%\begin{figure}[!ht]
%\includegraphics[width=\columnwidth]{./figs/Trig/pg5.eps}
%\caption{}
%\label{fig:trig_pg5}
%\end{figure}
%%
%\\
%\solution Fig. \ref{fig:trig_pg5} summarizes the problem. The objective is to find $h$.  from the figure,
%%
%\begin{align}
%b_1 &= h\cot 60\degree
%\\
%b_2 &= h\cot 30\degree
%\\
%b_2-b_1 &= 40
%\\
%\implies h \brak{\cot 30\degree-\cot 60\degree}&= 40
%\\
%\text{or } h &= \frac{40}{\cot 30\degree-\cot 60\degree}
%\end{align}
%
\item The angles of depression of the top and the bottom of an 8m tall building from the top of a multi-storeyed building are 30$\degree$ and 45$\degree$ respectively.  Find the height of the multi-storeyed building and the distance between the two buildings.
%
%\begin{figure}[!ht]
%\includegraphics[width=\columnwidth]{./figs/Trig/pg6.eps}
%\caption{}
%\label{fig:trig_pg6}
%\end{figure}
%%
%\\
%\solution Fig. \ref{fig:trig_pg6} summarizes the problem. The objective is to find $h_2$ and $b$.  From the figure, 
%%
%\begin{align}
%h_2 &= b\tan \theta_2
%\\
%h_2-h_1 &= b\tan \theta_1
%\end{align}
%%
%which can be expressed as
%%
%\begin{align}
%\myvec{
% 1 & -\tan\theta_2 
%\\
% 1 & -\tan\theta_1
%}\myvec{h_2\\b}
%= h_1\myvec{0\\1}
%\end{align}
%%
%and solved.
\item A traffic signal board, indicating ‘SCHOOL AHEAD’, is an equilateral triangle with side 'a'. Find the area of the signal board, using Heron's formula. If its perimeter is 180 cm, what will be the area of the signal board?
	\iffalse
\item The triangular side walls of a flyover have been used for advertisements. The sides of the walls are 122 m, 22 m and 120 m. The advertisements yield an earning of \rupee 5000 per $m^2$ per year.  A company hired one o its walls for 3 months. How  much rent did it pay?
	\fi
\item There is a slide in a park. One of its side walls has been painted in some colour with a message ``KEEP THE PARK GREEN AND CLEAN". If the sides of the wall are 15 m, 11 m and 6 m, find the area painted in colour.
\item Find the area of a triangle two sides of which are 18cm and 10cm and the perimeter is 42cm.
\item Sides of a triangle are in the ratio of 12 : 17 : 25 and its perimeter is 540cm. Find its area. 
\item  An isosceles triangle has perimeter 30 cm and each of the equal sides is 12 cm. Find the area of the triangle.
\item A girl walks 4km west, then she walks 3km in a direction $30\degree$ east of north and stops.  Determine the girl's displacement from her initial point of departure.
%
\item A circus artist is climbing a 20m long rope, which is tightly stretched and tied from the top of a vertical pole to the ground.  Find the height of the pole, if the angle made by the rope with the ground level is 30$\degree$.
%
\item A tree breaks due to storm and the broken part bends so that the top of the tree touches the ground making an angle of 30$\degree$ with it.  The distance between the foot of the tree to the point where the top touches the ground is 8m.  Find the height of the tree.
%
\item A contractor plans to install two slides for the children to play in a park.  For the children below the age of 5 years, she prefers to have a slide whose top is at a height of 1.5m, and is inclined at an angle of 30$\degree$  to the ground, whereas for elder children she wants to have a steep slide at a height of 3m, and inclined at an angle of 60$\degree$ to the ground.  What should be the length of the slide in each case?
%
\item The angle of elevation of the top of a tower from a point on the ground, which is 30m away from the foot of the tower, is 30$\degree$.  Find the height of the tower.
%
\item A kite is flying at a height of 60m above the ground.  The string attached to the kite is temporarily tied to a point on the ground.  The inclination of the string with the ground is $60\degree$.  Find the length of the string, assuming that there is no slack in the string.
%
\item A 1.5m tall boy is standing at some distance from a 30m tall building.  The angle of elevation from his eyes to the top of the building increases from 30$\degree$
 to 60$\degree $ as he walks towards the building.  Find the distance he walked towards the building.

\item From a point on the ground, the angles of elevation of the bottom and the top of a transmission tower fixed at the top of a 20 m high building are 45$\degree$ and 60$\degree$ respectively. Find the height of the tower.

\item A statue, 1.6 m tall, stands on the top of a pedestal. From a point on the ground, the angle of elevation of the top of the statue is 60$\degree$ and from the same point the angle of elevation of the top of the pedestal is 45$\degree$. Find the height of the pedestal.
\item The angle of elevation of the top of a building from the foot of the tower is 30$\degree$ and the angle of elevation of the top of the tower from the foot of the building is 60$\degree$. If the tower is 50 m high, find the height of the building.
\item Two poles of equal heights are standing opposite each other on either side of the road, which is 80 m wide. From a point between them on the road, the angles of elevation of the top of the poles are 60$\degree$ and 30$\degree$, respectively. Find the height of the poles and the distances of the point from the poles.
\item A TV tower stands vertically on a bank of a canal. From a point on the other bank directly opposite the tower, the angle of elevation of the top of the tower is 60$\degree$. From another point 20 m away from this point on the line joing this point to the foot of the tower, the angle of elevation of the top of the tower is 30$\degree$. Find the height of the tower and the width of the canal.
\item From the top of a 7 m high building, the angle of elevation of the top of a cable tower is 60$\degree$ and the angle of depression of its foot is 45$\degree$. Determine the height of the tower.
\item As observed from the top of a 75 m high lighthouse from the sea-level, the angles of depression of two ships are 30$\degree$ and 45$\degree$. If one ship is exactly behind the other on the same side of the lighthouse, find the distance between the two ships.
\item A 1.2 m tall girl spots a balloon moving with the wind in a horizontal line at a height of 88.2 m from the ground. The angle of elevation of the balloon from the eyes of the girl at any instant is 60$\degree$. After some time, the angle of elevation reduces to 30$\degree$. Find the distance travelled by the balloon during the interval.
\item A straight highway leads to the foot of a tower. A man standing at the top of the tower observes a car at an angle of depression of 30$\degree$, which is approaching the foot of the tower with a uniform speed. Six seconds later, the angle of depression of the car is found to be 60$\degree$. Find the time taken by the car to reach the foot of the tower from this point.
\item The angles of elevation of the top of a tower from two points at a distance of 4 m and 9 m from the base of the tower and in the same straight line with it are complementary. Prove that the height of the tower is 6 m.
%
\item $E$ and $F$ are points on the sides  $PQ$  and PR respectively of a  $\triangle PQR$ . For each of the following cases, state whether $EF  \parallel  QR$ 
\begin{enumerate}
\item  $PE = 3.9 cm, EQ = 3 cm, PF = 3.6 cm$ and $FR = 2.4 cm $
\item  $PE = 4 cm, QE = 4.5 cm, PF = 8 cm$ and $RF = 9 cm $
\item   $PQ  = 1.28 cm, PR = 2.56 cm, PE = 0.18 cm$ and $PF = 0.36 cm$
\end{enumerate}
\item A girl of height 90 cm is walking away from the base of a lamp-post at a speed of 1.2 m/s. If the lamp is 3.6 m above the ground, find the length of her shadow after 4 seconds.
\item  $ \triangle  ODC \sim  \triangle  OBA, \angle BOC = 125 \degree$ and $\angle CDO = 70 \degree$. Find $\angle DOC, \angle DCO$ and $\angle OAB$.
\item  Nazima is fly fishing in a stream. The tip of her fishing rod is 1.8 m above the surface of the water and the fly at the end of the string rests on the water 3.6 m away and 2.4 m from a point directly under the tip of the rod. Assuming that her string (from the tip of her rod to the fly) is taut, how much string does she have out? If she pulls in the string at the rate of 5 cm per second, what will be the horizontal distance of the fly from her after 12 seconds?
%
\item  A vertical pole of length 6 m casts a shadow 4 m long on the ground and at the same time a tower casts a shadow 28 m long. Find the height of the tower.
\item Let  $\triangle  ABC  \sim   \triangle  DEF$ and their areas be, respectively, 64 $cm^2$ and 121 $cm^2$.  If $EF = 15.4 cm$, find BC.
\item A ladder is placed against a wall such that its foot is at a distance of 2.5 m from the wall and its top reaches a window 6 m above the ground. Find the length of the ladder.
\item Sides of triangles are given below. Determine which of them are right triangles. In case of a right triangle, write the length of its hypotenuse. 
\begin{enumerate}
\item  7 cm, 24 cm, 25 cm 
\item  3 cm, 8 cm, 6 cm 
\item  50 cm, 80 cm, 100 cm 
\item  13 cm, 12 cm, 5 cm
\end{enumerate}
\item  A ladder 10 m long reaches a window 8 m above the ground. Find the distance of the foot of the ladder from base of the wall.
\item  A guy wire attached to a vertical pole of height 18 m is 24 m long and has a stake attached to the other end. How far from the base of the pole should the stake be driven so that the wire will be taut?
\item  An aeroplane leaves an airport and flies due north at a speed of 1000 km per hour. At the same time, another aeroplane leaves the same airport and flies due west at a speed of 1200 km per hour. How far apart will be the two planes after $1\frac{1}{2}$ hours?
\item  Two poles of heights 6 m and 11 m stand on a plane ground. If the distance between the feet of the poles is 12 m, find the distance between their tops.
\item  In  $\triangle  ABC, AB = 6\sqrt{3} cm, AC = 12 cm$ and $BC = 6 cm$. Find the angle $B$.
%
\item A park, in the shape of a quadrilateral $ABCD$, has $\angle C = 90\degree, AB = 9 m, BC = 12 m, CD = 5 m$ and $ AD = 8 m$. How much area does it occupy?
2. Find the area of a quadrilateral $ABCD$ in which $AB = 3 cm, BC = 4 cm, CD = 4 cm, DA = 5 cm$ and $AC = 5 cm$.
\item A triangle and a parallelogram have the same base and the same area. If the sides of the triangle are 26 cm, 28 cm and 30 cm, and the parallelogram stands on the base 28 cm, find the height of the parallelogram.
\item A rhombus shaped field has green grass for 18 cows to graze. If each side of the rhombus is 30 m and its longer diagonal is 48 m, how much area of grass field will each cow be getting?
\item A field is in the shape of a trapezium whose parallel sides are 25 m and 10 m. The non-parallel sides are 14 m and 13 m. Find the area of the field.
%
\item $ABCD$ is a parallelogram, $AE  \perp  DC$ and $CF  \perp  AD$. If $AB = 16 cm$, $AE = 8$ cm and $CF = 10$ cm, find $AD$.
\item Kamla has a triangular field with sides 240 m, 200 m, 360 m, where she grew wheat. In another triangular field with sides 240 m, 320 m, 400 m adjacent to the previous field, she wanted to grow potatoes and onions. She divided the field in two parts by joining the mid-point of the longest side to the opposite vertex and grew patatoes in one part and onions in the other part. Draw the figure for this problem.  How much area (in hectares) has been used for wheat, potatoes and onions? (1 hectare = 10000 $m^2$).
\item Students of a school staged a rally for cleanliness campaign. They walked through the lanes in two groups. One group walked through the lanes AB, BC and CA; while the other through AC, CD and DA. Then they cleaned the area enclosed within their lanes. If AB = 9 m, BC = 40 m, CD = 15 m, DA = 28 m and $\angle B = 90\degree$, which group cleaned more area and by how much? Draw the corresponding figure.  Find the total area cleaned by the students (neglecting the width of the lanes). 
%
\item Sanya has a piece of land which is in the shape of a rhombus. She wants her one daughter and one son to work on the land and produce different crops. She divided the land in two equal parts. If the perimeter of the land is 400 m and one of the diagonals is 160 m, how much area each of them will get for their crops? Draw the rhombus.
%
\item Three girls Reshma, Salma and Mandip are playing a game by standing on a circle of radius 5m drawn in a park. Reshma throws a ball to Salma, Salma to Mandip, Mandip to Reshma. If the distance between Reshma and Salma and between Salma and Mandip is 6m each, what is the distance between Reshma and Mandip?
\item A circular park of radius 20m is situated in a colony. Three boys Ankur, Syed and David are sitting at equal distance on its boundary each having a toy telephone in his hands to talk each other. Find the length of the string of each phone.
    \item The longest side of a triangle is 3 times the shortest side and the third side is 2 cm shorter than the longest side. If the perimeter of the triangle is at least 61 cm, find the minimum length of the shortest side.
\item A rectangular park is to be designed whose breadth is 3 m less than its length. Its area is to be 4 square metres more than the area of a park that has already been made in the shape of an isosceles triangle with its base as the breadth of the rectangular park and of altitude 12 m. Find its length and breadth.
\item The area of a rectangular plot is 528 $m^2$
. The length of the plot (in metres) is one more than twice its breadth. We need to find the length and breadth of the plot.
%
\item  The altitude of a right triangle is 7 cm less than its base. If the hypotenuse is 13 cm, find the other two sides.
%
\item The diagonal of a rectangular field is 60 metres more than the shorter side. If the longer side is 30 metres more than the shorter side, find the sides of the field.
\item Is it possible to design a rectangular mango grove whose length is twice its breadth, and the area is 800 $m^2$
? If so, find its length and breadth.
%
\item Is it possible to design a rectangular park of perimeter 80 m and area 400 $m^2$ If so, find  its length and breadth.
\item On an open ground, a motorist follows a track that turns to his left by an angle of 600 after every 500 m. Starting from a given turn, specify the displacement of the motorist at the third, sixth and eighth turn. Compare the magnitude of the displacement with the total path length covered by the motorist in each case.
\item A passenger arriving in a new town wishes to go from the station to a hotel located 10 km away on a straight road from the station. A dishonest cabman takes him along a circuitous path 23 km long and reaches the hotel in 28 min. What is 
\begin{enumerate}
\item  the average speed of the taxi, 
\item  the magnitude of average velocity ? Are the two equal ?

\end{enumerate}
\item An aircraft is flying at a height of 3400 m above the ground. If the angle subtended at a ground observation point by the aircraft positions 10.0 s apart is 30$\degree$, what is the speed of the aircraft ?
	\iffalse
\item Two identical billiard balls strike a rigid wall with the same speed but at different angles, and get reflected without any change in speed, as shown in Fig. \ref{fig:5.6}. What is 
\begin{enumerate}
\item  the direction of the force on the wall due to each ball? 
\item the ratio of the magnitudes of impulses imparted to the balls by the wall ?
\end{enumerate}
\begin{figure}[!ht]
\centering
\includegraphics[width=\columnwidth]{./figs/11-1/5/5.6.eps}
\caption{}
\label{fig:5.6}
\end{figure} 

\item See Fig. \ref{fig:5.8}. A mass of 6 kg is suspended by a rope of length 2 m from the ceiling. A force of 50 N in the horizontal direction is applied at the midpoint P of the rope, as shown. What is the angle the rope makes with the vertical in equilibrium ? (Take g = 10 $m s^{-2}$).
\begin{figure}[!ht]
\centering
\includegraphics[width=\columnwidth]{./figs/11-1/5/5.8.eps}
\caption{}
\label{fig:5.8}
\end{figure} 
\item See Fig. \ref{fig:5.11}. A mass of 4 kg rests on a horizontal plane. The plane is gradually inclined until at an angle $\theta = 15\degree$ with the horizontal, the mass just begins to slide. What is the coefficient of static friction between the block and the surface ?
\begin{figure}[!ht]
\centering
\includegraphics[width=\columnwidth]{./figs/11-1/5/5.11.eps}
\caption{}
\label{fig:5.11}
\end{figure} 
\item A cyclist speeding at 18 km/h on a level road takes a sharp circular turn of radius 3 m without reducing the speed. The co-efficient of static friction between the tyres and the road is 0.1. Will the cyclist slip while taking the turn?
\item A circular racetrack of radius 300 m is banked at an angle of 15$\degree$. If the coefficient of friction between the wheels of a race-car and the road is 0.2, what is the 
\begin{enumerate}
\item  optimum speed of the racecar to avoid wear and tear on its tyres, and 
\item  maximum permissible speed to avoid slipping ?
\end{enumerate}
\item An aircraft executes a horizontal loop at a speed of 720 km/h with its wings banked at 15$\degree$. What is the radius of the loop ?
\item A train runs along an unbanked circular track of radius 30 m at a speed of 54 km/h. The mass of the train is $10^6$
kg. What provides the centripetal force required for this
purpose - The engine or the rails ? What is the angle of banking required to prevent wearing out of the rail ?
\item A disc revolves with a speed of $33 \frac{1}{ 3}$ rev/min 
 and has a radius of 15 cm. Two coins are
placed at 4 cm and 14 cm away from the centre of the record. If the co-efficient of friction between the coins and the record is 0.15, which of the coins will revolve with the record ?
\item A 70 kg man stands in contact against the inner wall of a hollow cylindrical drum of radius 3 m rotating about its vertical axis with 200 rev/min. The coefficient of friction between the wall and his clothing is 0.15. What is the minimum rotational speed of the cylinder to enable the man to remain stuck to the wall (without falling) when the floor is suddenly removed ?
\item A thin circular loop of radius R rotates about its vertical diameter with an angular frequency $\omega$. Show that a small bead on the wire loop remains at its lowermost point for $\omega \le \sqrt{\frac{g}{R}}$ . What is the angle made by the radius vector joining the centre to the bead with the vertical downward direction for $\omega = \sqrt{\frac{2g}{R}}$  ? Neglect friction.
\item A stone of mass 0.25 kg tied to the end of a string is whirled round in a circle of radius 1.5 m with a speed of 40 rev./min in a horizontal plane. What is the tension in the string ? What is the maximum speed with which the stone can be whirled around if the string can withstand a maximum tension of 200 N ?
\item A woman pushes a trunk on a railway platform which has a rough surface. She applies a force of 100 N over a distance of 10 m. Thereafter, she gets progressively tired and her applied force reduces linearly with distance to 50 N. The total distance through which the trunk has been moved is 20 m. Plot the force applied by the woman and the frictional force, which is 50 N versus displacement. Calculate the work done by the two forces over 20 m.
\item A bob of mass m is suspended by a light string of length L . It is imparted a horizontal velocity $v_o$
such that it completes a semi-circular trajectory in the vertical plane with the string becoming slack only on reaching the topmost point, C. This is shown in Fig. \ref{fig:6.6}. Obtain an expression for 
\begin{enumerate}
\item  vo; 
\item  the speeds at points B and C; 
\item  the ratio of the kinetic energies ($K_B/K_C$) at B and C. 
\end{enumerate}
Comment on the nature
of the trajectory of the bob after it reaches the point C.
\begin{figure}[!ht]
\includegraphics[width=\columnwidth]{./figs/11-1/6/6.6.eps}
\caption{}
\label{fig:6.6}
\end{figure}
\item The bob A of a pendulum released from 30\degree to the
vertical hits another bob B of the same mass at rest on a table as shown in Fig. \ref{fig:6.15}. How high does the bob A rise after the collision? Neglect the size of the bobs and assume the collision to be elastic.
\begin{figure}[!ht]
\includegraphics[width=\columnwidth]{./figs/11-1/6/6.15.eps}
\caption{}
\label{fig:6.15}
\end{figure}
%
\item The bob of a pendulum is released from a horizontal position. If the length of the pendulum is 1.5 m, what is the speed with which the bob arrives at the lowermost point, given that it dissipated 5\% of its initial energy against air resistance ?
\item The blades of a windmill sweep out a circle of area A. 
\begin{enumerate}[label=(\alph*)]
\item  If the wind flows at a velocity v perpendicular to the circle, what is the mass of the air passing through it in time t ? 
\item  What is the kinetic energy of the air ? 
\item  Assume that the windmill converts 25\% of the wind’s energy into electrical energy, and that $A = 30 m^2, v = 36km/h$ and the density of air is $1.2 kg m^{-3}$. What is the electrical power produced ?
\end{enumerate}
\item A bullet of mass 0.012 kg and horizontal speed $70 m s^{-1}$.
strikes a block of wood of
mass 0.4 kg and instantly comes to rest with respect to the block. The block is suspended from the ceiling by means of thin wires. Calculate the height to which the block rises. Also, estimate the amount of heat produced in the block.
\item Two inclined frictionless tracks, one gradual and the other steep meet at A from where two stones are allowed to slide down from rest, one on each track (Fig. \ref{fig:6.16}). Will the stones reach the bottom at the same time? Will they reach there with the same speed? Explain. Given $\theta_1 = 30\degree\theta_2 = 60\degree$  and $h = 10m$, what are the speeds and times taken by the two stones?
\begin{figure}[!ht]
\includegraphics[width=\columnwidth]{./figs/11-1/6/6.16.eps}
\caption{}
\label{fig:6.16}
\end{figure}
\item  A 1 kg block situated on a rough incline is connected to a spring of spring constant 100 N m–1
as shown in Fig. \ref{fig:6.17}. The block is released from rest with the spring in the
unstretched position. The block moves 10 cm down the incline before coming to rest. Find the coefficient of friction between the block and the incline. Assume that the
spring has a negligible mass and the pulley is frictionless.
\begin{figure}[!ht]
\includegraphics[width=\columnwidth]{./figs/11-1/6/6.17.eps}
\caption{}
\label{fig:6.17}
\end{figure}
\item Find the centre of mass of three particles at the vertices of an equilateral triangle. The masses of the particles are 100g, 150g, and 200g respectively. Each side of the equilateral triangle is 0.5m long.
\item Find the centre of mass of a triangular lamina.
\item Find the centre of mass of a uniform L-shaped lamina (a thin flat plate) with dimensions as shown in Fig. \ref{fig:7.11}. The mass of the lamina is 3 kg.
\begin{figure}[!ht]
\includegraphics[width=\columnwidth]{./figs/11-1/7/7.11.eps}
\caption{}
\label{fig:7.11}
\end{figure}
\item A metal bar 70 cm long and 4.00 kg in mass supported on two knife-edges placed 10 cm from each end. A 6.00 kg load is suspended at 30 cm from one end. Find the reactions at the knifeedges. (Assume the bar to be of uniform cross section and homogeneous.)
\item A 3m long ladder weighing 20 kg leans on a frictionless wall. Its feet rest on the floor 1 m from the wall as shown in Fig. \ref{fig:7.27}. Find the reaction forces of the wall and the floor.
\begin{figure}[!ht]
\includegraphics[width=\columnwidth]{./figs/11-1/7/7.27.eps}
\caption{}
\label{fig:7.27}
\end{figure}
\item What is the moment of inertia of a rod of mass M, length l about an axis perpendicular to it through one end?
\item What is the moment of inertia of a ring about a tangent to the circle of the ring?
\item  A cord of negligible mass is wound round the rim of a fly wheel of mass 20 kg and radius 20 cm. A steady pull of 25 N is applied on the cord as shown in Fig. \ref{fig:7.35}. The flywheel is mounted on a horizontal axle with frictionless bearings.
\begin{figure}[!ht]
\includegraphics[width=\columnwidth]{./figs/11-1/7/7.35.eps}
\caption{}
\label{fig:7.35}
\end{figure}
\begin{enumerate}[label=(\alph*)]
\item  Compute the angular acceleration of the wheel.
\item  Find the work done by the pull, when 2m of the cord is unwound.
\item  Find also the kinetic energy of the wheel at this point. Assume that the wheel starts from rest.
\item  Compare answers to parts (b) and (c).
\end{enumerate}
\item Three bodies, a ring, a solid cylinder and a solid sphere roll down the same inclined plane without slipping. They start from rest. The radii of the bodies are identical. Which of the bodies reaches the ground with maximum velocity?
	\iffalse
\item In the HCl molecule, the separation between the nuclei of the two atoms is about 
1.27 \si{\angstrom} (1 \si{\angstrom} = $10^{-10}$ m). 
Find the approximate location of the CM of the molecule, given that a chlorine atom is about 35.5 times as massive as a hydrogen atom and nearly all the mass of an atom is concentrated in its nucleus.
\fi
\item A non-uniform bar of weight W is suspended at rest by two strings of negligible weight as shown in Fig. \ref{fig:7.39}. The angles made by the strings with the vertical are 36.9$\degree$ and 53.1$\degree$ respectively. The bar is 2 m long. Calculate the distance d of the centre of gravity of the bar from its left end.
\begin{figure}[!ht]
\includegraphics[width=\columnwidth]{./figs/11-1/7/7.39.eps}
\caption{}
\label{fig:7.39}
\end{figure}
\item A car weighs 1800 kg. The distance between its front and back axles is 1.8 m. Its centre of gravity is 1.05 m behind the front axle. Determine the force exerted by the level ground on each front wheel and each back wheel.
\item 
\begin{enumerate}[label=(\alph*)]
\item Find the moment of inertia of a sphere about a tangent to the sphere, given the moment of inertia of the sphere about any of its diameters to be $\frac{2MR^2}{5}$ where M is the mass of the sphere and R is the radius of the sphere.
\item  Given the moment of inertia of a disc of mass M and radius R about any of its diameters to be $\frac{MR^2}{4}$, find its moment of inertia about an axis normal to the
disc and passing through a point on its edge. 
\end{enumerate}
\item Torques of equal magnitude are applied to a hollow cylinder and a solid sphere, both having the same mass and radius. The cylinder is free to rotate about its standard axis of symmetry, and the sphere is free to rotate about an axis passing through its centre. Which of the two will acquire a greater angular speed after a given time.
\item A solid cylinder of mass 20 kg rotates about its axis with angular speed 100 $rad s^{-1}$. The radius of the cylinder is 0.25 m. What is the kinetic energy associated with the rotation of the cylinder? What is the magnitude of angular momentum of the cylinder about its axis?
\item 
\begin{enumerate}[label=(\alph*)]

\item  A child stands at the centre of a turntable with his two arms outstretched. The turntable is set rotating with an angular speed of 40 rev/min. How much is the angular speed of the child if he folds his hands back and thereby reduces his moment of inertia to 2/5 times the initial value ? Assume that the turntable rotates without friction.
\item Show that the child's new kinetic energy of rotation is more than the initial kinetic energy of rotation. How do you account for this increase in kinetic energy?
\end{enumerate}
\item  A rope of negligible mass is wound round a hollow cylinder of mass 3 kg and radius 40 cm. What is the angular acceleration of the cylinder if the rope is pulled with a force of 30 N ? What is the linear acceleration of the rope ? Assume that there is no slipping.
\item  To maintain a rotor at a uniform angular speed of 200 $rad s^{-1}$, an engine needs to transmit a torque of 180 N m. What is the power required by the engine ? (Note: uniform angular velocity in the absence of friction implies zero torque. In practice, applied torque is needed to counter frictional torque). Assume that the engine is 100\% efficient.
\item  From a uniform disk of radius R, a circular hole of radius R/2 is cut out. The centre of the hole is at R/2 from the centre of the original disc. Locate the centre of gravity of the resulting flat body.
\item  A metre stick is balanced on a knife edge at its centre. When two coins, each of mass 5 g are put one on top of the other at the 12.0 cm mark, the stick is found to be balanced at 45.0 cm. What is the mass of the metre stick?
\item A solid sphere rolls down two different inclined planes of the same heights but different angles of inclination. 
\begin{enumerate}[label=(\alph*)]
\item  Will it reach the bottom with the same speed in each case? 
\item  Will it take longer to roll down one plane than the other? 
\item  If so, which one and why?
\end{enumerate}
\item  A hoop of radius 2 m weighs 100 kg. It rolls along a horizontal floor so that its centre of mass has a speed of 20 cm/s. How much work has to be done to stop it? 
\item  The oxygen molecule has a mass of $5.30  \times  10^{-26} $ kg and a moment of inertia of
$1.94 \times 10^{-46} kg m^2$ about an axis through its centre perpendicular to the lines joining
the two atoms. Suppose the mean speed of such a molecule in a gas is 500 m/s and that its kinetic energy of rotation is two thirds of its kinetic energy of translation. Find the average angular velocity of the molecule.
\item  A solid cylinder rolls up an inclined plane of angle of inclination 30$\degree$. At the bottom of the inclined plane the centre of mass of the cylinder has a speed of 5 m/s. 
\begin{enumerate}[label=(\alph*)]
\item How far will the cylinder go up the plane? 
\item  How long will it take to return to the bottom?
\end{enumerate}
\item  As shown in Fig. \ref{fig:7.40}, the two sides of a step ladder BA and CA are 1.6 m long and hinged at A. A rope DE, 0.5 m is tied half way up. A weight 40 kg is suspended from a point F, 1.2 m from B along the ladder BA. Assuming the floor to be frictionless and neglecting the weight of the ladder, find the tension in the rope and forces exerted by the floor on the ladder. (Take $g = 9.8 m/s^2$
) (Hint: Consider the equilibrium of each side of the ladder separately.) 
\begin{figure}[!ht]
\includegraphics[width=\columnwidth]{./figs/11-1/7/7.40.eps}
\caption{}
\label{fig:7.40}
\end{figure}
%
\item A man stands on a rotating platform, with his arms stretched horizontally holding a 5 kg weight in each hand. The angular speed of the platform is 30 revolutions per minute. The man then brings his arms close to his body with the distance of each weight from the axis changing from 90cm to 20cm. The moment of inertia of the man together with the platform may be taken to be constant and equal to 7.6 $kg m^2$.
\begin{enumerate}[label=(\alph*)]
\item  What is his new angular speed? (Neglect friction.) 
\item  Is kinetic energy conserved in the process? If not, from where does the change come about?
\end{enumerate}
\item A bullet of mass 10 g and speed 500 m/s is fired into a door and gets embedded exactly at the centre of the door. The door is 1.0 m wide and weighs 12 kg. It is hinged at one end and rotates about a vertical axis practically without friction. Find the angular speed of the door just after the bullet embeds into it. (Hint: The moment of inertia of the door about the vertical axis at one end is $\frac{ML^2}{3}$.)
\item  Two discs of moments of inertia $I_1$ and $I_2$ about their respective axes (normal to the disc and passing through the centre), and rotating with angular speeds $\omega_1$ and $\omega_2$ are brought into contact face to face with their axes of rotation coincident. 
\begin{enumerate}[label=(\alph*)]
\item  What is the angular speed of the two-disc system? 
\item  Show that the kinetic energy of the combined system is less than the sum of the initial kinetic energies of the two discs. How do you account for this loss in energy? Take $\omega_1 \ne \omega_2$.
\end{enumerate}

\item Prove the result that the velocity v of translation of a rolling body (like a ring, disc, cylinder or sphere) at the bottom of an inclined plane of a height h is given by
\begin{align}
v^2 = \frac{2gh}{1+\frac{k^2}{R^2}}
\end{align}
using dynamical consideration (i.e. by consideration of forces and torques). Note k is the radius of gyration of the body about its symmetry axis, and R is the radius of the body. The body starts from rest at the top of the plane.
\item A disc rotating about its axis with angular speed $\omega_0$ is placed lightly (without any translational push) on a perfectly frictionless table. The radius of the disc is R. What are the linear velocities of the points A, B and C on the disc shown in Fig. \ref{fig:7.41}? Will the disc roll in the direction indicated ?
\begin{figure}[!ht]
\includegraphics[width=\columnwidth]{./figs/11-1/7/7.41.eps}
\caption{}
\label{fig:7.41}
\end{figure}
\item Explain why friction is necessary to make the disc in Fig. \ref{fig:7.41} roll in the direction indicated. 
\begin{enumerate}[label=(\alph*)]
\item  Give the direction of frictional force at B, and the sense of frictional torque, before perfect rolling begins.
\item  What is the force of friction after perfect rolling begins ?
\end{enumerate}
\item  A solid disc and a ring, both of radius 10 cm are placed on a horizontal table simultaneously, with initial angular speed equal to 10 $\pi rad s^{-1}$ . Which of the two will start to roll earlier ? The coefficient of kinetic friction is $\mu_k$
= 0.2.
\item  A cylinder of mass 10 kg and radius 15 cm is rolling perfectly on a plane of inclination 30\degree. The coefficient of static friction $\mu_s$ = 0.25. 
\begin{enumerate}[label=(\alph*)]
\item  How much is the force of friction acting on the cylinder ? 
\item  What is the work done against friction during rolling ? 
\item  If the inclination $\theta$ of the plane is increased, at what value of $\theta$ does the cylinder begin to skid, and not roll perfectly ?
\end{enumerate}
\item Three equal masses of m kg each are fixed at the vertices of an
equilateral triangle ABC. 
\begin{enumerate}[label=(\alph*)]
\item  What is the force acting on a mass 2m placed at the centroid G of the triangle? 
\item  What is the force if the mass at the vertex A is doubled ? Take AG = BG = CG = 1 m (see Fig. \ref{fig:8.5})
\end{enumerate}
\begin{figure}[!ht]
\includegraphics[width=\columnwidth]{./figs/11-1/8/8.5.eps}
\caption{}
\label{fig:8.5}
\end{figure}
\item Find the potential energy of a system of four particles placed at the vertices of a square of side $l$. Also obtain the potential at the centre of the square.
\item Two uniform solid spheres of equal radii R, but mass M and 4 M have a centre to centre separation 6 R, as shown in Fig. \ref{fig:8.10}. The two spheres are held fixed. A projectile of mass m is projected from the surface of the sphere of mass M directly towards the centre of the second sphere. Obtain an expression for the minimum speed v of the projectile so that it reaches the surface of the second sphere.
\begin{figure}[!ht]
\includegraphics[width=\columnwidth]{./figs/11-1/8/8.10.eps}
\caption{}
\label{fig:8.10}
\end{figure}
\fi
\end{enumerate}

