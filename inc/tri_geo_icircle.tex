%%
\renewcommand{\theequation}{\theenumi}
\begin{enumerate}[label=\arabic*.,ref=\thesubsection.\theenumi]
\numberwithin{equation}{enumi}

\item In Fig. \ref{fig:tri_icentre}, points $\vec{D}, \vec{E}, \vec{F}$  are at a distance $r$ from $\vec{I}$.  The circle with centre $\vec{I}$ through these points is known as the {\em incircle}. Draw the incircle of $\triangle ABC$.
%
\\
\solution This is done by the following python code
%
\begin{lstlisting}
codes/circle/tri_icircle.py
\end{lstlisting}
%
and the equivalent latex-tikz code to draw Fig. \ref{fig:tri_icircle} is
%
\begin{lstlisting}
figs/triangle/tri_icircle.tex
\end{lstlisting}

%
\item Sides $AB, BC$ and $CA$ touch the circle at exactly one point.  Such lines are known as {\em tangents} to the circle.
\item Tangents to the circle are perpendicular to the radius at the point of contact.
\item From \eqref{eq:tri_icentre_baudhd}, it is  obvious that tangents to the circle from a given point are equal.
%

\end{enumerate}

