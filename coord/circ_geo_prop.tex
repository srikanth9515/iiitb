%\subsection{Properties}
%
%\renewcommand{\theequation}{\theenumi}
\begin{enumerate}[label=\thesection.\arabic*.,ref=\thesection.\theenumi]
\numberwithin{equation}{enumi}
\item 
		In \figref{fig:circ_tang},  $OC$ is the radius and $PC$ touches the circle at $C$.  Show that	
  \begin{align}
	  OC \perp PC.
		\label{eq:circ_tang-line-orth}	
  \end{align}
	\begin{figure}[!ht]
		\begin{center}
			
			%\includegraphics[width=\columnwidth]{./figs/fig:circ_tang_icept}
			%\vspace*{-10cm}
			\resizebox{\columnwidth}{!}{\input{./figs/coord/circ_tang.tex}}
		\end{center}
		\caption{}
		\label{fig:circ_tang}	
	\end{figure}
	\\
		\solution
		The equation of $PC$ can be expressed as
  \begin{align}
		\label{erq:circ_tang-line}	
	  \vec{x} = 
	  \vec{C} + \mu\vec{m}
  \end{align}
  and the equation of the circle is 
  \begin{align}
		\label{eq:circ_tang-eq}	
	  \norm{\vec{x}-\vec{O}} = R
  \end{align}
  Substituting
		\eqref{erq:circ_tang-line}	
		in 
		\eqref{eq:circ_tang-eq},
  \begin{align}
	  \norm{\vec{C} + \mu\vec{m}- \vec{O}}^2 &= R^2
	  \\
	  \implies	  \mu^2 \norm{ \vec{m}}^2 
	  + 2\mu\vec{m}^{\top}\brak{\vec{C} -\vec{O}}
	  + \norm{\vec{C}-\vec{O}}^2 - R^2 &= 0
  \end{align}
  The above equation has only one root.  Hence the discriminant of the above quadratic should be zero. So, 
  \begin{align}
	  \cbrak{\vec{m}^{\top}\brak{\vec{C} -\vec{O}}}^2-\norm{ \vec{m}}^2 
	  \cbrak{ \norm{\vec{C}-\vec{O}}^2 - R^2} &= 0
		\label{erq:circ_tang-line-quad}	
  \end{align}
  Since $\vec{C}$ is a point on the circle, 
  \begin{align}
	   \norm{\vec{C}-\vec{O}}^2 - R^2 = 0
	   \\
	   \implies 
	  \vec{m}^{\top}\brak{\vec{C} -\vec{O}} = 0
		\label{eq:circ_tang-line-orth-dir}	
  \end{align}
upon substituting in \eqref{erq:circ_tang-line-quad}. Using the definition of the direction vector from 	
\eqref{eq:dir-vec}
  \begin{align}
	  \vec{m}=\vec{P}-\vec{C} 
	\\
	  \implies 
	  \brak{\vec{P}-\vec{C}}^{\top}\brak{\vec{C} -\vec{O}} = 0
		\label{eq:circ_tang-line-orth-pc}
  \end{align}
		which is equivalent to 
		\eqref{eq:circ_tang-line-orth}.

\item
In Fig. \ref{fig:circ_tang_icept} show that 
%
\begin{equation}
\theta = \alpha
		\label{fig:circ_tang_icept-equal}	
\end{equation}
%
\label{them:tang_icept_ang}
	\begin{figure}[!ht]
		\begin{center}
			
			%\includegraphics[width=\columnwidth]{./figs/fig:circ_tang_icept}
			%\vspace*{-10cm}
			\resizebox{\columnwidth}{!}{\input{./figs/coord/circ_tang_icept.tex}}
		\end{center}
		\caption{$\theta= \alpha$.}
		\label{fig:circ_tang_icept}	
	\end{figure}
	%
	\\
	\solution
	Let 
    Let
  \begin{align}
	  \vec{O} = \vec{0}
	  \vec{A} = \myvec{\cos \theta_1 \\ \sin \theta_1},\,
	  \vec{B} =  \myvec{\cos \theta_2 \\ \sin \theta_2},\,
	  \vec{C} =  \myvec{\cos \theta_3 \\ \sin \theta_3}
  \end{align}
  \iffalse
  From 
\eqref{eq:tri_ccentre_subtend-temp}
  \begin{align}
	  \cos \alpha = \cos\brak{ \frac{\theta_1-\theta_3}{2}  }
		\label{eq:circ_tang-line-alpha}	
  \end{align}
  \fi
  Without loss of generality,  let 
  \begin{align}
	  \theta_3 = \frac{\pi}{2}
		\label{eq:circ_tang-line-t3}	
  \end{align}
  Then, 
  \begin{align}
	  \vec{C}-\vec{O} = \myvec{0 \\ 1}
  \end{align}
  %which is the direction vector of the $y$-axis.
  From 
		from \eqref{eq:circ_tang-line-orth-pc},	
  \begin{align}
	  \vec{C}-\vec{P} = \myvec{1 \\ 0}
		\label{eq:circ_tang-line-pc},	
  \end{align}
From   
\eqref{eq:tri_cos_form-ccentre}
and 
		\eqref{eq:circ_tang-line-pc},	
  \begin{align}
	  \cos \theta &= \frac{
		  \myvec{\cos \theta_3-\cos \theta_1 & \sin \theta_3-\sin \theta_1}
		  \myvec{1 \\ 0}
		  }
		  {
	   2 \sin \brak{\frac{\theta_1-\theta_3}{2}}
			  } 
			  \\
			  &=
	    \sin \brak{\frac{\theta_1+\theta_3}{2}}
	    =\cos\brak{\frac{\pi}{2}-\frac{\theta_1+\theta_3}{2}}
	    =\cos\brak{\frac{\pi}{4}-\frac{\theta_1}{2}}
  \end{align}
  upon substituting from 
		\eqref{eq:circ_tang-line-t3}.  Similarly, 	
		from
\eqref{eq:tri_ccentre_subtend-temp},
  \begin{align}
	  \cos \alpha = \cos \brak{\frac{\theta_1-\theta_3}{2}  }
	    =\cos\brak{\frac{\pi}{4}-\frac{\theta_1}{2}}
	  =\cos \theta
  \end{align}
%
\iffalse
Using Theorem \ref{them:tang_icept_ang},
\begin{equation}
\triangle PAC \sim \triangle PBC \quad (AAA)
\end{equation}
 Hence,
%
\begin{align}
\frac{PA}{PC} &= \frac{PC}{PB} \\
\implies PA.PB &=PC^2
\end{align}
%
%\item
%Given that $PA.PB = PC^2$, show that $PC$ is a tangent to the circle.
%
%
\item
	In Fig. \ref{fig:chord_tang_prod}, show that
\begin{equation}
	PA.PB = PC.PD
	\end{equation}
%
\begin{figure}[!ht]
	\begin{center}
		
		%\includegraphics[width=\columnwidth]{./figs/fig:chord_tang_prod}
		%\vspace*{-10cm}
		\resizebox{\columnwidth}{!}{\input{./figs/misc/chord_tang_prod.tex}}
	\end{center}
	\caption{$PA.PB = PC^2$.}
	\label{fig:chord_tang_prod}	
\end{figure}
\\
\solution From Theorem \ref{them:circ_tang_icept_prod}, if $PT$ be a tangent to the circle, 	
\begin{equation}
	PA.PB = PT^2 =PC.PD
	\end{equation}
	\fi
\end{enumerate}
%
%\item
%	Fig. \ref{ch4_circle_def} represents a circle, which passes through the vertices $A B, C$ of  $\triangle ABC$ in Fig.  	\eqref{ch3_perp_bisector}	
% The points in the circle are at a distance $R$ from the {\em centre} $O$.  $R$ is known as the {\em radius}. The line  joining any two points on a circle is known as a {\em chord}.  Thus, the sides of $\triangle ABC$ are chords.
%
%\begin{figure}[!ht]
%	\begin{center}
%		
%		%\includegraphics[width=\columnwidth]{./figs/ch4_circle_def}
%		%\vspace*{-10cm}
%%		\resizebox{\columnwidth}{!}{\input{./figs/fig_4.0.tex}}
%		\resizebox{\columnwidth}{!}{\input{./figs/circumcircle.tex}}
%%		\resizebox{\columnwidth}{!}{\input{./figs/fig_4.0.tex}}
%	\end{center}
%	\caption{Circle Definitions}
%	\label{ch4_circle_def}	
%\end{figure}
%\item
%	In Fig. \ref{ch4_circle_def}, $A$ and $B$ are points on the circle.  The line $AB$ is known as a chord of the circle.

%
%
%\item
%	\label{ch4_prob_circle_subtend}
%	In Fig. \ref{ch4_circle_subtend}  
%%Show that $\angle AOB = 2\angle ACB $.
%
%\begin{figure}[!ht]
%	\begin{center}
%		
%		%\includegraphics[width=\columnwidth]{./figs/ch4_circle_subtend}
%		%\vspace*{-10cm}
%		\resizebox{\columnwidth}{!}{\input{./figs/circumcircle.tex}}
%%		\resizebox{\columnwidth}{!}{\input{./figs/fig_4.1.tex}}
%	\end{center}
%	\caption{Angle subtended by chord $AB$ at the centre $O$ is twice the angle subtended at $P$. }
%	\label{ch4_circle_subtend}	
%\end{figure}

%\solution In Fig. \ref{ch4_circle_subtend}, the triangleles $OPA$ and $OPB$ are isosceles. Hence,
%%
%\begin{align}
%\angle OCA = \angle OAC &= \theta_1 \\
%\angle OCB = \angle OBC &= \theta_2
%\end{align}
%%
%Also, $\alpha$ and $\beta$ are exterior angles corresponding to the triangle $AOC$ and $BOC$ respectively. Hence
%%
%\begin{align}
%\alpha &= 2\theta_1 \\
%\beta &= 2\theta_2
%\end{align}
%%
%Thus,
%%
%\begin{align}
%\angle AOB &= \alpha + \beta \\
%&= 2\brak{\theta_1 + \theta_2} \\
%&= 2\angle ACB
%\end{align}
%

%
\iffalse
\item
In Fig. \ref{fig:circ_dia_rt}, $AB$ is the diameter and passes through the centre $O$.  show that $\angle APB = 90^{\degree}$ .

%
\begin{figure}[!ht]
	\begin{center}
		
		%\includegraphics[width=\columnwidth]{./figs/fig:circ_dia_rt}
		%\vspace*{-10cm}
		\resizebox{\columnwidth}{!}{\input{./figs/misc/circ_dia_rt.tex}}
	\end{center}
	\caption{Diameter of a circle.}
	\label{fig:circ_dia_rt}	
\end{figure}
%
%\\
\solution From Theorem \ref{them:tri_ccentre_subtend}, 
\begin{align}
\angle APB = \frac{1}{2}\angle AOB = 90\degree
\end{align}
%
\item In  right $\triangle APB$, right angled at $\vec{P}$, the median 
\begin{align}
PO = AO = OB
\end{align}
%
\solution See Fig. \ref{fig:circ_dia_rt}. The median $PO$ is a radius of the circle.
\item
	In Fig. \ref{fig:circ_chord_prod}, show that 
	\begin{equation}
	PA.PB = PC.PD
	\end{equation}
\begin{figure}[!ht]
	\begin{center}
		
		%\includegraphics[width=\columnwidth]{./figs/fig:circ_chord_prod}
		%\vspace*{-10cm}
		\resizebox{\columnwidth}{!}{\input{./figs/misc/circ_chord_prod.tex}}
	\end{center}
	\caption{$PA.PB = PC.PD$}
	\label{fig:circ_chord_prod}	
\end{figure}

%
\solution From Theorem \ref{them:tri_ccentre_subtend}, 
	\begin{equation}
	\begin{split}
\angle ABD &= \angle ACD \\
\angle CAB &= \angle CDB	
	\end{split}
	\end{equation}
Hence, 
%
\begin{align}
\triangle PAC \sim \triangle  PBD \quad (AAA)
\end{align}
%
and
%
\begin{align}
\frac{PA}{PD} &= \frac{PC}{PB} \\
\implies PA.PB &= PC.PD
\end{align}
%
%\item
%	Fig. \ref{fig:incirc_def} touches the sides of $\triangle ABC$ \eqref{ch3_angle_bisector} and is known as  the {\em incircle}.  The sides of the $\triangle$	are known as the {\em tangents} of the circle.
%
%\begin{figure}[!ht]
%	\begin{center}
%		
%		%\includegraphics[width=\columnwidth]{./figs/ch4_circle_def}
%		%\vspace*{-10cm}
%%		\resizebox{\columnwidth}{!}{\input{./figs/fig_4.0.tex}}
%		\resizebox{\columnwidth}{!}{\input{./figs/incirc.tex}}
%%		\resizebox{\columnwidth}{!}{\input{./figs/fig_4.0.tex}}
%	\end{center}
%	\caption{Incircle and Tangent}
%	\label{fig:incirc_def}	
%\end{figure}
%%
%\item Tangents to a circle from any point outside the circle are equal.
%\\
%\solution See Fig. \ref{fig:incirc_def} and use \eqref{eq:tang_eq}.
%
%\item
%	Show that 
%	\begin{equation}
%	\label{ch5_sin_90}
%	\sin 90^{\degree} = 1
%	\end{equation}
%\solution From 	Prolem \ref{prob:ch2_triang_area}
%and Problem \ref{prob:tri_area_sin}, the area of the right angled $\triangle ABC$ in Fig. \ref{ch2_sq_ar} is 	
%	\begin{equation}
%	\label{ch5_sin_90_der}
%	\frac{1}{2}ab = \frac{1}{2}ab\sin 90^{\degree} 
%	\end{equation}
%%
%resulting in 	\eqref{ch5_sin_90}.
%\item
%	Show that 
%	\begin{equation}
%	\label{ch5_cos_90}
%	\cos 90^{\degree} = 0
%	\end{equation}
%
%\solution 
%Follows from the fact that $\sin 90\degree = 1$ and \eqref{eq:tri_sin_cos_id}.
%
%
%
%%
%% 	
%\item
%	The line $PX$ in Fig. \ref{ch4_tangent_def} touches the circle at exactly one  point $P$. 
%%It is known as the tangent to the circle.
%%
%%
%	Show that $OP \perp PX$.
%% is the perpendicular to the line $PX$ as shown in the Fig. \ref{ch4_short_dist}. Show that $OP$ is the shortest distance between the point $O$ and the line $PX$. 
%
%\solution Without loss of generality, let $0 \le \theta \le 90^{\degree}$. Using the cosine formula in $\triangle OPP_n$,\begin{align}
%\brak{r+d_n}^2 > r^2,
%\end{align}
%%Let $P_1$ be a point on the line $PX$. Then $OPP_1$ is a right angled triangle.  Using Budhayana's theorem,
%%
%\begin{figure}[!ht]
%	\begin{center}
%		
%		%\includegraphics[width=\columnwidth]{./figs/ch4_tangent_def}
%		%\vspace*{-10cm}
%		\resizebox{\columnwidth}{!}{\input{./figs/fig_4.6.tex}}
%	\end{center}
%	\caption{Tangent to a Circle.}
%	\label{ch4_tangent_def}	
%\end{figure}
%%
%%\begin{figure}[!ht]
%%	\begin{center}
%%		
%%		%\includegraphics[width=\columnwidth]{./figs/ch4_short_dist}
%%		%\vspace*{-10cm}
%%		\resizebox{\columnwidth}{!}{\input{./figs/fig_4.6_1.tex}}
%%	\end{center}
%%	\caption{Shortest distance from $O$ to line $PX$}
%%	\label{ch4_short_dist}	
%%\end{figure}
%
%%
%\begin{align}
%%\begin{split}
%\brak{r+d_n}^2 = r^2 + x_n^2 - 2rx_n\cos\theta > r^2 
%\\
%\implies  0 <\cos\theta < \frac{x_n}{2r},
%%OP_1^2 &= OP^2 + PP_1^2 \\
%%\implies OP_1 > OP
%%\end{split}
%\end{align}
%%
%where $x_n$ can be made as small as we choose.  Thus, 
%%
%\begin{align}
%\cos \theta = 0 \implies \theta  = 90 ^{\degree}
%\end{align}
%from \eqref{ch5_cos_90}.
%
%%\solution In Fig. \ref{ch4_tangent_def}, we can see that $OP$ is is the radius of the circle and the length of all line segments from $O$ to the line $PX > r$.  Using the result of the previous 
%problem, it is obvious that $OP \perp PX$. 
%
	%
\fi
